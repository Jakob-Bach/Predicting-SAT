\documentclass{article}

\title{Predicting Satisfiability of Benchmark Instances}
\author{Jakob Bach \and Markus Iser}

\usepackage[style=ieee, backend=bibtex]{biblatex}
\usepackage{graphicx} % plots
\usepackage{hyperref} % links and URLs
\addbibresource{references.bib}

\begin{document}

\maketitle

\begin{abstract}

\end{abstract}

\section{Introduction}
\label{sec:introduction}

\paragraph{Motivation}

SATzilla~\cite{xu2008satzilla, xu2012satzilla2012}, ISAC~\cite{kadioglu2010isac}, SNNAP~\cite{collautti2013snnap}

\paragraph{Contributions}

\paragraph{Results}

\paragraph{Outline}

Section~\ref{sec:related-work} reviews related work.
Section~\ref{sec:experimental-design} introduces our experimental design.
Section~\ref{sec:evaluation} presents experimental results.
Section~\ref{sec:conclusion} concludes.

\section{Related Work}
\label{sec:related-work}

\section{Experimental Design}
\label{sec:experimental-design}

\subsection{Datasets}

- SAT Competition 2020~\cite{balyo2020proceedings} main (400)
- SAT Competition 2021~\cite{balyo2021proceedings} main (400)
- all instances from `meta' with results known (11159)

Datasets may contain missing values.

Features (251):
- SATzilla~\cite{xu2012features} (138)
- base (56) + gate (57)

\subsection{Predictions}

random forests~\cite{breiman2001random} and XGBoost~\cite{chen2016xgboost}
Matthews Correlation Coefficient (MCC)~\cite{matthews1975comparison, gorodkin2004comparing}

\subsection{Implementation}

\emph{scikit-learn} \cite{pedregosa2011scikit}, \emph{gbd-tools}~\cite{iser2020collaborative}

\section{Evaluation}
\label{sec:evaluation}

\section{Conclusions and Future Work}
\label{sec:conclusion}

\subsection{Conclusions}

\subsection{Future Work}

\printbibliography

\end{document}
